\chapter{CAD development}
\section{Brief about CAD modelling}
The geometric kernel used in Osdag is OpenCascade where PythonOCC is considered to be the wrapper over the kernel. The modeling scheme used in OpenCascade is Boundary Representation (BRep) with geometric primitives as vertex, edge, face, body and prism. Operators used in BRep are called as Boolean operators with the boolean operations as add, subtract and union. This is what the fundamental used for CAD development in Osdag.

\section{Flow of methods/variables for CAD model}
As shown in the above figure, the python files in Component directory are backbone of CAD model in Osdag. Each python file in Component directory is a class file which returns the 3D CAD model of whatever dimensions passed to each python file.
viz. in a case of plate.py which includes a class object as Plate and it takes in inputs as length (L), width(W) and thickness (T) to create the CAD model of plate. Therefore, whatever dimensions you pass on in to this (plate.py) class file, subsequent CAD model of plate would get generated. Similar logic is applied to rest of the class files.
Now, we will look at how the data from one python file to another file passes to create 3D CAD model with one example.
The CAD model in Osdag is displayed, if and only if the design is safe else the graphics window stays dark.

\section{Generalized flow of parameters}
The generalized flow of variables/parameters is as discussed below;
\begin{enumerate}
	\item When user clicks on Design button, all entered inputs are validated and component's dimensions are calculated in calculation file. Once, this is done then compiler directs to function, \textit{def call\_3DModel(self)}
	\item Depending upon the safe or unsafe design it changes the graphics window color
	\item  In the same python file (Main\_File) compiler moves to that function which contain the name of connectivity viz. ColFlangeBeamWeb, ColWebBeamWeb, ExtendedBothWay etc.
	\item This function basically initializes parameters for CAD model by taking UiObj (User inputs) and OutputObj (Output dictionary).
	\item Next, the cursor moves to another python file with the same name as of connectivity name.
	\item This file creates CAD model by calling the respective class file from Component directory and returns the CAD model of individual component.
	
\end{enumerate}

\section{CAD for moment connection of extended both way}
The above mentioned steps we'll illustrate by taking an example of Both Way Extended End Plate in Moment Connection. The below mentioned steps will give you the complete CAD model including
welds at appropriate places.
\begin{enumerate}
	\item After user clicks on Design button in GUI, the debugger enters into the function named def design\_btnclicked() in main file
	\item For CAD model creation, reach on line def display\_3DModel(self,component, bgcolor)
	\item Depending upon safe or unsafe design the graphics window changes the background color. Consider, here as the design is safe.
	\item  Here, it calls the function named def createExtendedBothWays() for creating CAD model of connection.
	\item In first step here, beam properties are assigned from database depending upon the beam cross section selected by user.
	\item Plate dimensions are pulled from output dictionary.
	\item Bolt dimensions are decided on size of diameter selected by user.
	\item For nut and bolt placement, it calls to NutBoltArray() class in nutBoltPlacement.py file
	\item After getting all the data and it displays each model by get\_model() function.
\end{enumerate}
